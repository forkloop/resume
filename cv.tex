% LaTeX Curriculum Vitae
%


\documentclass[letterpaper, 10pt]{article}

%\usepackage{kvoptions}
\usepackage{hyperref}
\usepackage{geometry}
\usepackage{engord}
\usepackage{fancyhdr}
% Comment the following lines to use the default Computer Modern font
% instead of the Palatino font provided by the mathpazo package.
% Remove the 'osf' bit if you don't like the old style figures.
\usepackage[T1]{fontenc}
\usepackage[sc,osf]{mathpazo}
% \usepackage{caption}
% \renewcommand{\captionfont}{\large \bfseries \sffamily}
% Set your name here
\def\name{Xin Liu}

\hypersetup{
  colorlinks = true,
  urlcolor = black,
  pdfauthor = {\name},
  pdfkeywords = {computer science, telecommunications, mathematics},
  pdftitle = {\name: Curriculum Vitae},
  pdfsubject = {Curriculum Vitae},
  pdfpagemode = UseNone
}

\geometry{
  body={6.8in, 9.5in},
  left=0.8in,
  top=1in
}

% Customize page headers
\pagestyle{myheadings}
\markright{\name}
\thispagestyle{fancy}

% Custom section fonts
\usepackage{sectsty}
\sectionfont{\rmfamily\mdseries\Large}
\subsectionfont{\rmfamily\mdseries\itshape\large}

% Don't indent paragraphs.
\setlength\parindent{0em}

% Make lists without bullets
\renewenvironment{itemize}{
  \begin{list}{}{
    \setlength{\leftmargin}{1.5em}
  }
}{
  \end{list}
}

\newcommand{\resitem}[1]{\item #1 \vspace{-2pt}}
% \newcommand{\resheading}[1]{{\large \parashade[.9]{sharpcorners}{\textbf{#1 \vphantom{p\^{E}}}}}}
\newcommand{\ressubheading}[4]{
\begin{tabular*}{6.6in}{l@{\extracolsep{\fill}}r}
		#1 & #2 \\
		\textit{#3} & \textit{#4} \\
\end{tabular*}\vspace{-6pt}}

\cfoot{}
\begin{document}
% Place name at left
{\huge \name}

\vspace{0.25in}
\begin{minipage}{0.65\textwidth}
Dept. of Computer Science \& Engineering\\
\href{http://www.buffalo.edu}{University at Buffalo}\\
%Homepage: \href{http://www.cse.buffalo.edu/~xliu36}{http://www.cse.buffalo.edu/\~ xliu36}
\end{minipage}
\begin{minipage}{0.45\textwidth}\vspace{-0.1in}
\begin{tabular}{ll}
Phone: &806-789-7269 \\
Email: &\href{mailto:forkloop@gmail.com}{\tt forkloop@gmail.com} \\\end{tabular}
\end{minipage}
Homepage: \href{http://www.cse.buffalo.edu/~xliu36}{http://www.cse.buffalo.edu/$\small{\sim}$xliu36}

%\vspace{-0.5cm}
\section*{Education}
\begin{itemize}
\item
\ressubheading{University at Buffalo The State University of New York}{}{M.S., Computer Science and Engineering (expected 2013)}
{Sept. 2011 - present}

\item
\ressubheading{Nanjing University of Posts and Telecommunications(NUPT)}{}{B.E., Telecommunication Engineering}
{Sept. 2007 - Jun. 2011}
\end{itemize}

\section*{Experience}
\begin{itemize}
\item \textbf{SDE Intern at Amazon.com}
\hspace{3.4in} \textit{May. 2012 - Aug. 2012}
\begin{itemize}
\item{I worked on a project that migrate the data transformation from data warehouse to EMR with Hadoop and Hive.}
\end{itemize}
\end{itemize}

\vspace{-0.2in}
\section*{Projects}
\begin{itemize}
\item \textbf{Sparse Coding with Message Passing Interface}

\hspace{5.2in} \textit{Nov. 2011 - Dec. 2011}
\begin{itemize}
\item{Sparse coding is an effective unsupervised learning method that can find a succinct representation for various inputs. However, the
expensive computational cost has hindered its application for large-scale data. Here we parallel sparse coding with message passing interface (MPI). Based on the two recursive iterations for basis learning and coefficient learning, by averaging the computational cost over 16 computers with MPI, we can achieve a 5.x speedup over the sequential method.
}
\end{itemize}
\item \textbf{Automated Epileptic Diagnosis based on {EEG} series}

\hspace{5.2in} \textit{Apr. 2010 - Aug. 2011}
% \vspace{-10pt}
\begin{itemize}
\item{We apply the methods of Machine Learning to diagnose epilepsy automatically. First we extract different features
from {EEG/MEG} series of both epilepsy patients and controls and select a most discriminative feature set, then train different classifiers with
this feature set. I also collaborate into coding a Python module - {PyEEG} for the extraction of {EEG/MEG} features. {PyEEG} available at 
\textit{http://code.google.com/p/pyeeg/}}
\end{itemize}

%\item \textbf{Registering SAM field information with MRI structural information for surgical navigation}
%
%\hspace{13.5cm}\textit{May.2010 - Sept.2010}
%% \vspace{-10pt}
%\begin{itemize}
%\item{We developed a free software named as~{CTF-SAM-OUT} that can register and superpose the Synthetic aperture 
%magnetometry (SAM) field information produced by CTF MEG\texttrademark system on Magnetic resonance imaging(MRI) generated by 
%GE\texttrademark system. With this new higher resolution MRI image registered with SAM information shown on surgical navigation, doctors can 
%perform the surgery more efficiently and effectively. Software is available at \textit{http://sourceforge.net/projects/ctfsamout/}}
%\end{itemize}
\end{itemize}

\vspace{-0.5cm}
\section*{Interests \& Skills}
\begin{itemize}
\item Familiar with: \quad Python, Ruby, Java, C, Android, Hadoop, Full Stack Web Development (Ruby on Rails).
\item Interested in: \quad Node.js, Machine Learning, Parallel Computing(MPI, OpenMP)
%\item Interested areas: \quad Computer Vision, Machine Learning, Biomedical signal analysis.
\end{itemize}

\section*{Honors \& Awards}
%  \begin{tabular}{ll}
\begin{tabular*}{6.5in}{ll@{\extracolsep{\fill}}r}\vspace{0.2cm}
 \engordnumber{1} & prize in  \href{http://en.mcm.edu.cn/}{Contemporary Undergraduate Mathematical Contest in Modeling, China.} (1.9\%) & 2009\\
 \engordnumber{1} & prize in   ACM Programming Contest of School of Telecommunications & \\\vspace{0.2cm}
 & \& Information Engineering, NUPT. & 2008\\\vspace{0.2cm}
 \engordnumber{2} &  prize in Mathematical Contest in Modeling, NUPT. &  2009\\\vspace{0.2cm}
 \engordnumber{2} & prize in Science and Technology Innovation Training Program, NUPT. & 2009\\\vspace{0.2cm} 
\engordnumber{2} & prize in "ZTE Corporation Electronic Design Cup" Competition, NUPT. &  2008\\\vspace{0.2cm}
 \engordnumber{3} & Scholarship, NUPT. (25\%) &  2010 - 2007\\\vspace{0.2cm}
\end{tabular*}


\vspace{-0.5cm}
\section*{Publications}
\begin{itemize}
\item \textbf{Xin Liu} and Ying Ding, \textit{General Scaled Support Vector Machines}, accepted by~\textit{3rd International Conference on 
Machine Learning and Computing, Singapore, 2011}. ~\textit{http://arxiv.org/abs/1009.5268}
\item Forrest Sheng Bao, \textbf{Xin Liu} and Christina Zhang, \textit{PyEEG: An Open Source Python Module for EEG/MEG Feature Extraction}, 
\textit{Computational Intelligence and Neuroscience.}, 1-7, 2011.
\end{itemize}



\end{document}
